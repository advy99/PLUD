\documentclass[12pt, spanish]{article}
\usepackage[spanish]{babel}
\selectlanguage{spanish}
%\usepackage{natbib}
\usepackage{url}
\usepackage[utf8]{inputenc}
\usepackage{graphicx}
\graphicspath{{images/}}
\usepackage{parskip}
\usepackage{fancyhdr}
\usepackage{vmargin}
\usepackage{multirow}
\usepackage{float}
\usepackage{chngpage}

\usepackage{amsfonts}

\usepackage{caption}
\usepackage{subcaption}

\usepackage{hyperref}
\usepackage[
    type={CC},
    modifier={by-nc-sa},
    version={4.0},
]{doclicense}

\hypersetup{
    colorlinks=true,
    linkcolor=blue,
    filecolor=magenta,
    urlcolor=cyan,
}

% para codigo
\usepackage{listings}
\usepackage{xcolor}



%% configuración de listings

\definecolor{listing-background}{HTML}{F7F7F7}
\definecolor{listing-rule}{HTML}{B3B2B3}
\definecolor{listing-numbers}{HTML}{B3B2B3}
\definecolor{listing-text-color}{HTML}{000000}
\definecolor{listing-keyword}{HTML}{435489}
\definecolor{listing-identifier}{HTML}{435489}
\definecolor{listing-string}{HTML}{00999A}
\definecolor{listing-comment}{HTML}{8E8E8E}
\definecolor{listing-javadoc-comment}{HTML}{006CA9}

\lstdefinestyle{eisvogel_listing_style}{
  language         = python,
%$if(listings-disable-line-numbers)$
%  xleftmargin      = 0.6em,
%  framexleftmargin = 0.4em,
%$else$
  numbers          = left,
  xleftmargin      = 0em,
 framexleftmargin = 0em,
%$endif$
  backgroundcolor  = \color{listing-background},
  basicstyle       = \color{listing-text-color}\small\ttfamily{}\linespread{1.15}, % print whole listing small
  breaklines       = true,
  frame            = single,
  framesep         = 0.19em,
  rulecolor        = \color{listing-rule},
  frameround       = ffff,
  tabsize          = 4,
  numberstyle      = \color{listing-numbers},
  aboveskip        = 1.0em,
  belowskip        = 0.1em,
  abovecaptionskip = 0em,
  belowcaptionskip = 1.0em,
  keywordstyle     = \color{listing-keyword}\bfseries,
  classoffset      = 0,
  sensitive        = true,
  identifierstyle  = \color{listing-identifier},
  commentstyle     = \color{listing-comment},
  morecomment      = [s][\color{listing-javadoc-comment}]{/**}{*/},
  stringstyle      = \color{listing-string},
  showstringspaces = false,
  escapeinside     = {/*@}{@*/}, % Allow LaTeX inside these special comments
  literate         =
  {á}{{\'a}}1 {é}{{\'e}}1 {í}{{\'i}}1 {ó}{{\'o}}1 {ú}{{\'u}}1
  {Á}{{\'A}}1 {É}{{\'E}}1 {Í}{{\'I}}1 {Ó}{{\'O}}1 {Ú}{{\'U}}1
  {à}{{\`a}}1 {è}{{\'e}}1 {ì}{{\`i}}1 {ò}{{\`o}}1 {ù}{{\`u}}1
  {À}{{\`A}}1 {È}{{\'E}}1 {Ì}{{\`I}}1 {Ò}{{\`O}}1 {Ù}{{\`U}}1
  {ä}{{\"a}}1 {ë}{{\"e}}1 {ï}{{\"i}}1 {ö}{{\"o}}1 {ü}{{\"u}}1
  {Ä}{{\"A}}1 {Ë}{{\"E}}1 {Ï}{{\"I}}1 {Ö}{{\"O}}1 {Ü}{{\"U}}1
  {â}{{\^a}}1 {ê}{{\^e}}1 {î}{{\^i}}1 {ô}{{\^o}}1 {û}{{\^u}}1
  {Â}{{\^A}}1 {Ê}{{\^E}}1 {Î}{{\^I}}1 {Ô}{{\^O}}1 {Û}{{\^U}}1
  {œ}{{\oe}}1 {Œ}{{\OE}}1 {æ}{{\ae}}1 {Æ}{{\AE}}1 {ß}{{\ss}}1
  {ç}{{\c c}}1 {Ç}{{\c C}}1 {ø}{{\o}}1 {å}{{\r a}}1 {Å}{{\r A}}1
  {€}{{\EUR}}1 {£}{{\pounds}}1 {«}{{\guillemotleft}}1
  {»}{{\guillemotright}}1 {ñ}{{\~n}}1 {Ñ}{{\~N}}1 {¿}{{?`}}1
  {…}{{\ldots}}1 {≥}{{>=}}1 {≤}{{<=}}1 {„}{{\glqq}}1 {“}{{\grqq}}1
  {”}{{''}}1
}
\lstset{style=eisvogel_listing_style}


\usepackage[default]{sourcesanspro}

\setmarginsrb{2 cm}{1 cm}{2 cm}{2 cm}{1 cm}{1.5 cm}{1 cm}{1.5 cm}

\title{Entrega final:\\
Estado actual del proyecto.\hspace{0.05cm} }
\author{Guillermo Sandoval Schmidt\\ Antonio David Villegas Yeguas}
\date{\today}

\makeatletter
\let\thetitle\@title
\let\theauthor\@author
\let\thedate\@date
\makeatother

\pagestyle{fancy}
\fancyhf{}
\rhead{\theauthor}
\lhead{\thetitle}
\cfoot{\thepage}

\begin{document}

%%%%%%%%%%%%%%%%%%%%%%%%%%%%%%%%%%%%%%%%%%%%%%%%%%%%%%%%%%%%%%%%%%%%%%%%%%%%%%%%%%%%%%%%%

\begin{titlepage}
    \centering
    \vspace*{0.3 cm}
    \includegraphics[scale = 0.50]{ugr.png}\\[0.7 cm]
    %\textsc{\LARGE Universidad de Granada}\\[2.0 cm]
    \textsc{\large 4º CSI 2020/21}\\[0.5 cm]
    \textsc{\large Grado en Ingeniería Informática}\\[0.5 cm]
    \rule{\linewidth}{0.2 mm} \\[0.2 cm]
    { \huge \bfseries \thetitle}\\
    \rule{\linewidth}{0.2 mm} \\[1 cm]

    \begin{minipage}{0.4\textwidth}
        \begin{flushleft} \large
            \emph{Autores:}\\
            \theauthor\\
            \end{flushleft}
            \end{minipage}~
            \begin{minipage}{0.4\textwidth}
            \begin{flushright} \large
            \emph{Asignatura: \\
            Programación Lúdica}   \\
        \end{flushright}
    \end{minipage}\\[0.5cm]

    {\large \thedate}\\[0.5cm]
    %{\url{https://github.com/advy99/AA/}}
    {\doclicenseThis}

    \vfill

\end{titlepage}

%%%%%%%%%%%%%%%%%%%%%%%%%%%%%%%%%%%%%%%%%%%%%%%%%%%%%%%%%%%%%%%%%%%%%%%%%%%%%%%%%%%%%%%%%

\tableofcontents
\pagebreak

%%%%%%%%%%%%%%%%%%%%%%%%%%%%%%%%%%%%%%%%%%%%%%%%%%%%%%%%%%%%%%%%%%%%%%%%%%%%%%%%%%%%%%%%%

\section{Integrantes del grupo}

Los integrantes del grupo somos:

\begin{itemize}
	\item Guillermo Sandoval Schmidt
	\item Antonio David Villegas Yeguas
\end{itemize}

\section*{Estudio previo}
\label{sec:estudio_previo}
\addcontentsline{toc}{section}{\nameref{sec:estudio_previo}}

\section{Propuesta}

\subsection{Título provisional}

Como título provisional hemos decidido \textbf{Lose to Win}, debido a la principal mecánica del videojuego que explicaremos en la descripción.

\subsection{Descripción del juego, género y público objetivo}


Lose to Win es un juego multijugador de partidas rápidas basado en minijuegos, donde los jugadores se enfrentan todos contra todos o por equipos para alzarse con la victoria.

Cuenta con una mecánica transversal a los diversos minijuegos, ya que para ganar debes perder el minijuego.

Algunos de los minijuegos que hemos pensado son:

\begin{enumerate}
	\item Bomb tag: uno de los jugadores aparece con una bomba, al tocar al jugador con la bomba se la robas. Gana el jugador que tenga la bomba cuando explote.
	\item Egg Giver: el clásico juego de coger huevos y llevarlos a tu base pero al revés, gana el jugador que menos huevos tenga en su base.
	\item Spike Ball: similar al minijuego de “Move or Die” de mismo nombre, pero el objetivo es que la bola de pinchos te mate, ganando el jugador que más veces muera.
	\item Generous Guy: recoge objetos del color del rival para que gane puntos y ganar siendo el jugador con menos puntos.
\end{enumerate}


Nuestra idea no se encaja en un único género, debido a la gran variedad de minijuegos y objetivos de las distintas partidas, aun así, los principales géneros de nuestra idea serían acción, party game y plataformas.

En principio el público objetivo sería cualquier persona, independientemente de la edad (PEGI 3), aunque es cierto que esto variará dependiendo del estilo gráfico del juego y como de explícito sea con respecto a la violencia de los minijuegos, tanto a nivel de que tipos de minijuegos como a nivel artístico, lo que conllevaría un PEGI 12.

\section{Estudio de mercado}

Aunque hemos encontrado una gran cantidad de videojuegos similares a nuestra idea, finalmente hemos escogido estos tres ya que son los más parecidos a nuestra propuesta.

\subsection{Move or Die}

Se trata de un videojuego para cuatro jugadores en el que cada jugador controla un personaje cuya vida se reduce rápidamente si el jugador deja de moverse por un momento, se regenerá si reanuda el movimiento. Cada ronda contará con distintas reglas o modificadores, que varían entre los distintos modos. El desafío surge cuando los jugadores siguen moviéndose para ganar, mientras evitan peligros como picos o bloques que caen. Los jugadores también pueden empujarse entre sí.


\subsubsection{Detalles sobre el videojuego}

\begin{itemize}
	\item \textbf{Compañia}: Those Awesome Guys
	\item \textbf{Plataformas}: Windows, Linux, macOS, PlayStation 4
	\item \textbf{Modelo de negocio}: Juego de pago (Pay-to-play)
	\item \textbf{Web}: \url{http://www.moveordiegame.com}
\end{itemize}

\subsubsection{Capturas del videojuego}

\begin{figure}[H]
  \centering
   \includegraphics[width=\textwidth]{"competencia/mod_juego.jpg"}
	\caption{Imagen de Move or Die dentro de una partida.}
\end{figure}

\begin{figure}[H]
  \centering
   \includegraphics[width=\textwidth]{"competencia/mod_editor.jpg"}
	\caption{Editor de niveles de Move or Die.}
\end{figure}

\begin{figure}[H]
  \centering
   \includegraphics[width=\textwidth]{"competencia/mod_lobby.jpg"}
	\caption{Lobby de Move or Die.}
\end{figure}

\subsubsection{Aspectos positivos}

\begin{itemize}
	\item Juego local y online.
	\item De 1 a 4 jugadores.
	\item Incluye editor de niveles.
	\item Facilidad para crear mods.
	\item Compatible con mando y teclado.
	\item Juego multiplataforma (puedes jugar con una persona que no esté jugando en tu misma plataforma).
	\item Variedad de minijuegos.
\end{itemize}

\subsubsection{Aspectos negativos}

\begin{itemize}
	\item Requiere muchas horas desbloquear los distintos modos de juego.
	\item Modos de juego bloqueados que solo se pueden desbloquear con “monedas del juego”.
	\item Malos servidores para jugar online.
\end{itemize}

\subsection{Party Panic}

Es un juego multijugador de hasta cuatro jugadores en el que compiten unos contra otros en distintos minijuegos. Gana el jugador que más minijuegos complete victorioso.

\subsubsection{Detalles sobre el videojuego}

\begin{itemize}
	\item \textbf{Compañia}: Everglow Interactive
	\item \textbf{Plataformas}: Windows, macOS, Linux, XBOX ONE, PlayStation 4
	\item \textbf{Modelo de negocio}: Juego de pago (Pay-to-play)
	\item \textbf{Web}: \url{https://partypanicgame.com/}
\end{itemize}

\subsubsection{Capturas del videojuego}


\begin{figure}[H]
  \centering
   \includegraphics[width=\textwidth]{"competencia/pp_juego.jpg"}
	\caption{Imagen de Party Panic dentro de una partida.}
\end{figure}

\begin{figure}[H]
  \centering
   \includegraphics[width=\textwidth]{"competencia/pp_juego2.jpg"}
	\caption{Imagen de Party Panic dentro de una partida.}
\end{figure}


\begin{figure}[H]
  \centering
   \includegraphics[width=\textwidth]{"competencia/pp_lobby.jpg"}
	\caption{Lobby de Party Panic.}
\end{figure}

\subsubsection{Aspectos positivos}

\begin{itemize}
	\item Juego local y online.
	\item De 1 a 4 jugadores.
	\item Compatible con mando y teclado.
\end{itemize}

\subsubsection{Aspectos negativos}

\begin{itemize}
	\item No tiene editor de niveles ni soporte de mods.
	\item La IA no funciona correctamente.
	\item Interfaz muy básica y poco accesible.
	\item En una partida no se permiten jugadores de distintas plataformas.
	\item Falta variedad de jugadores.
	\item Malos servidores.
\end{itemize}

\subsection{Duck Game}

Es un videojuego de acción donde los personajes son patos que incluye características de videojuegos de disparos y plataformas. El juego tiene un modo de un jugador al igual que un modo multijugador en línea con un límite de hasta siete jugadores más. En este modo de juego el jugador que reciba un solo disparo muere y el jugador sobreviviente gana la ronda.


\subsubsection{Detalles sobre el videojuego}

\begin{itemize}
	\item \textbf{Compañia}: Adult Swim Games
	\item \textbf{Plataformas}: Windows, PlayStation 4, Switch
	\item \textbf{Modelo de negocio}: Juego de pago (Pay-to-play)
	\item \textbf{Web}: \url{https://www.adultswim.com/games/duck-game}
\end{itemize}

\subsubsection{Capturas del videojuego}

\begin{figure}[H]
  \centering
   \includegraphics[width=\textwidth]{"competencia/dg_juego.jpg"}
	\caption{Imagen de una partida de Duck Game.}
\end{figure}

\begin{figure}[H]
  \centering
   \includegraphics[width=\textwidth]{"competencia/dg_muestra.jpg"}
	\caption{Imagen de una escena de Duck Game.}
\end{figure}


\begin{figure}[H]
  \centering
   \includegraphics[width=\textwidth]{"competencia/dg_lobby.jpg"}
	\caption{Lobby de Duck Game.}
\end{figure}

\subsubsection{Aspectos positivos}

\begin{itemize}
	\item Juego local y online.
	\item De 2 a 8 jugadores.
	\item Incluye editor de niveles.
	\item Compatible con mando y teclado.
	\item En una partida no se permiten jugadores de distintas plataformas.
\end{itemize}

\subsubsection{Aspectos negativos}

\begin{itemize}
	\item Malos servidores.
	\item No permite modificar el esquema de control.
	\item No permite utilizar múltiples mandos en juego local.
	\item No permite permite varios jugadores
\end{itemize}

\section*{Estado actual}
\label{sec:estado_actual}
\addcontentsline{toc}{section}{\nameref{sec:estado_actual}}

\section{Resumen del estado actual}

Actualmente el proyecto se encuentra en un estado previo al lanzamiento, donde todas las ideas que planteamos están finalizadas, pero es necesario realizar unas pruebas en profundidad de cara a calibrar y mejorar la experiencia final de usuario.

Presentamos un juego multijugador competitivo, que actualmente se ejecuta localmente pero con la idea de llevarlo a un juego en red. Lose To Win cuenta con unos controles e interfaces simples e intuitivas de cara a que cualquier persona con solo mirar la pantalla sea capaz de jugar, teniendo una curva de aprendizaje nula y permitiendo saber jugar y divertirse desde el primer momento.

En Lose To Win jugaras partidas rápidas y frenéticas, compuestas por diversos minijuegos muy simples, donde el objetivo será perder en el minijuego, invirtiendo la forma tradicional de entender los juegos.

Actualmente contamos con tres minijuegos completados, pudiendo jugar en modo práctica o bien jugando a todos los minijuegos con el objetivos de llegar a lo alto de la tabla de puntuaciones globales. Con respecto a las opciones, contamos con diversas opciones para personalizar el idioma, sonido, pantalla y controles. En todas estas secciones entraremos en detalle más adelante.


\begin{figure}[H]
  \centering
   \includegraphics[width=\textwidth]{"menu_principal.png"}
	\caption{Menú principal de Lose To Win, con cuatro jugadores.}\label{figure:titulo}
\end{figure}


\section{Interfaz de usuario}

La interfaz de usuario se trata de una interfaz simple, dividida en dos secciones, la pantalla de juego, y una pantalla donde los usuarios podrán ver tanto sus controles y como añadir jugadores si se encuentran en un menú o su puntuación si se encuentran jugando:

\begin{figure}[H]
  \centering
	\includegraphics[width=\textwidth]{"interfaz/interfaz_inferior_anadir.png"}
  \caption{Interfaz inferior con dos jugadores en un menú.}\label{figure:interfaz_inferior_anadir}
\end{figure}

\begin{figure}[H]
  \centering
	\includegraphics[width=\textwidth]{"interfaz/interfaz_inferior_jugando.png"}
  \caption{Interfaz inferior con cuatro jugadores jugando.}\label{figure:interfaz_inferior_jugando}
\end{figure}

Con respecto al resto de interfaz, esta se encuentra totalmente integrada en el propio juego, los jugadores con sus personajes se moverán por plataformas que les llevarán a las distintas secciones disponibles, de forma que la adaptación a los controles del juego se hace de una forma escalada, en la que los jugadores pueden aprender a manejar a los personajes desde el principio, navegando por la interfaz. La única excepción será en el menú de opciones, donde la modificación de opciones se realizará con el ratón por motivos de comodidad frente al gran número de opciones disponibles.

De cara a poder navegar por el juego y sus menús será necesario que los jugadores se pongan de acuerdo sobre a que sección han de navegar, ya que el juego pedirá que todos los jugadores activos se encuentren en la misma plataforma para pasar de una sección a otra. Todas las plataformas que llevan a secciones están marcadas con símbolos en el mapa, además de tener un texto de acompañamiento junto a una cuenta atrás para asegurar que se trata del menú correcto:

\begin{figure}[H]
  \centering
	\includegraphics[width=\textwidth]{"interfaz/menu_no_funciona_num_jugadores.png"}
  \caption{El menú no se activa aunque dos jugadores quieran ir a esa sección, ya que otros dos no se han decidido.}\label{figure:menu_no_funciona_num_jugadores}
\end{figure}

\begin{figure}[H]
  \centering
	\includegraphics[width=\textwidth]{"interfaz/esperando_practicar.png"}
  \caption{Cuatro jugadores pasando a otro menú.}\label{figure:esperando_practicar}
\end{figure}

\begin{figure}[H]
  \centering
	\includegraphics[width=\textwidth]{"interfaz/menu_funciona_num_jugadores.png"}
  \caption{Cambiando el número de jugadores, seguimos necesitando a la totalidad de los jugadores activos.}\label{figure:menu_funciona_num_jugadores}
\end{figure}

Como vemos, a la vez que el jugador aprende a manejarse con su personaje, se mueve por los menús practicando el movimiento y saltos, a la vez de utilizar una interfaz simple y directa que deja claro en todo momento que es cada opción acompañándola de un mensaje.

\subsection{Secciones de la interfaz de usuario}

La interfaz de usuario está dividida de la siguiente forma:

\subsubsection{Menú principal}

Tras iniciar el juego nos encontraremos con el menú inicial, con el número mínimo de jugadores, dos:

\begin{figure}[H]
  \centering
	\includegraphics[width=\textwidth]{"interfaz/modo_inicial_defecto.png"}
  \caption{Menú inicial por defecto, con dos jugadores.}\label{figure:modo_inicial_defecto}
\end{figure}

En este menú podemos realizar cuatro opciones, de arriba a abajo y de izquierda a derecha:

\begin{enumerate}
	\item Ir al menú de juego.
	\item Ir al menú de opciones.
	\item Ir al menú de créditos.
	\item Salir del juego.
\end{enumerate}

\subsubsection{Menú de juego}

Al pasar al menú de juego, veremos lo siguiente:

\begin{figure}[H]
	\centering
	\includegraphics[width=\textwidth]{"modos/menu_juego.png"}
	\caption{Menú de juego.}\label{figure:menu_juego}
\end{figure}

Como vemos, contamos con cuatro opciones:

\begin{enumerate}
	\item Entrar a jugar.
	\item Ir al modo práctica.
	\item Ir a la tabla de puntuaciones.
	\item Volver al menú anterior.
\end{enumerate}


En la primera opción pasaremos al modo de juego por defecto, que explicaremos más adelante, en la segunda opción iremos al menú de práctica, donde podremos escoger entre los tres minijuegos disponibles actualmente y jugar de forma indefinida, con el objetivo de entender como funcionan y entrenarnos para jugar:

\begin{figure}[H]
	\centering
	\includegraphics[width=\textwidth]{"modos/menu_practica.png"}
	\caption{Menú de práctica.}\label{figure:menu_practica}
\end{figure}

Por último, en la tabla de puntuaciones podremos consultar quienes son los mejores jugadores de todo el juego, ya que como veremos más adelante en el modo de juego, tras acabar una partida el ganador podrá guardar su nombre con su puntuación en esta tabla y pasar a la historia de Lose To Win como el mayor perdedor:

\begin{figure}[H]
	\centering
	\includegraphics[width=\textwidth]{"tabla_puntuaciones.png"}
	\caption{Menú de puntuaciones.}\label{figure:tabla_puntuaciones}
\end{figure}


\subsubsection{Menú de opciones}

En el menú de opciones encontraremos una interfaz controlada por ratón para controlar las distintas configuraciones disponibles. También encontraremos dos plataformas, una a la izquierda para guardar la configuración y a la derecha para salir sin guardar los cambios realizados:

\begin{figure}[H]
	\centering
	\includegraphics[width=\textwidth]{"opciones/idiomas/spanish.png"}
	\caption{Menú de opciones.}\label{figure:menu_opciones}
\end{figure}


Con respecto a este menú y las opciones disponibles entraremos en detalle en la sección de opciones.

\subsubsection{Menú de créditos}

En este menú podremos encontrar una plataforma para volver al menú principal, además de un cuadro de texto con varias páginas que nos permitirá ver quien ha desarrollado este proyecto, así como dar crédito a todos los recursos utilizados para su desarrollo:

\begin{figure}[H]
	\centering
	\includegraphics[width=\textwidth]{"interfaz/creditos.png"}
	\caption{Menú de créditos.}\label{figure:creditos}
\end{figure}

\section{Opciones}

Con respecto a las opciones, hemos añadido distintas opciones de cara a que los usuarios puedan ajustar su modo de juego, desde el idioma, el nivel de la música y efectos de sonido, configurar opciones de la pantalla y personalizar los controles.

\begin{figure}[H]
    \centering
	 \begin{subfigure}[b]{0.49\textwidth}
		 \centering
		 \includegraphics[width=\textwidth]{"opciones/pagina1_opciones.png"}
		 \caption{Primera página del menú de opciones.}\label{fig:pagina1_opciones}
	 \end{subfigure}
	 \begin{subfigure}[b]{0.49\textwidth}
		 \centering
		\includegraphics[width=\textwidth]{"opciones/pagina2_opciones.png"}
		\caption{Segunda página del menú de opciones.}\label{fig:pagina2_opciones}
   \end{subfigure}
	\caption{Opciones disponibles en Lose To Win.}\label{fig:opciones}
\end{figure}

\subsection{Idioma}

Para introducir distintos idiomas hemos diseñado un sistema escalable, de forma que tenemos todas las lineas utilizadas almacenadas en ficheros que cualquier persona podría diseñar, y con un simple cambio en el código incorporar un nuevo idioma.

Actualmente contamos con seis traducciones:

\begin{enumerate}
	\item Español.
	\item Inglés.
	\item Alemán.
	\item Italiano.
	\item Francés.
	\item Portugués.
\end{enumerate}

Modificar el idioma es tan sencillo como pulsar un botón en el menú de juego, y el cambio se hará efectivo en el momento, aunque no permanecerá a no ser que guardemos las opciones:


\begin{figure}[H]
	\centering
	\includegraphics[width=\textwidth]{"opciones/idiomas/spanish.png"}
	\caption{Lose To Win en español.}\label{figure:spanish}
\end{figure}

\begin{figure}[H]
	\centering
	\includegraphics[width=\textwidth]{"opciones/idiomas/english.png"}
	\caption{Lose To Win en inglés.}\label{figure:english}
\end{figure}

\begin{figure}[H]
	\centering
	\includegraphics[width=\textwidth]{"opciones/idiomas/german.png"}
	\caption{Lose To Win en alemán.}\label{figure:german}
\end{figure}

\begin{figure}[H]
	\centering
	\includegraphics[width=\textwidth]{"opciones/idiomas/italian.png"}
	\caption{Lose To Win en italiano.}\label{figure:italian}
\end{figure}

\begin{figure}[H]
	\centering
	\includegraphics[width=\textwidth]{"opciones/idiomas/french.png"}
	\caption{Lose To Win en francés.}\label{figure:french}
\end{figure}

\begin{figure}[H]
	\centering
	\includegraphics[width=\textwidth]{"opciones/idiomas/portuguese.png"}
	\caption{Lose To Win en portugués.}\label{figure:portuguese}
\end{figure}


Además de cambiar todo lo mostrado en las imágenes, también cambiarán los textos de los menús, puntuaciones, etc, para adaptarse al idioma seleccionado.

\subsection{Sonido}

Para el sonido contamos con dos barras de selección, una para escoger el nivel de música y otra para el nivel de los efectos de sonido, y un botón para silenciar el juego por completo:

\begin{figure}[H]
	\centering
	\includegraphics[width=0.6\textwidth]{"opciones/sonido.png"}
	\caption{Opciones de sonido en Lose To Win.}\label{figure:sonido}
\end{figure}


Las barras de sonido, con el objetivo de ser más flexibles a la hora de seleccionar el nivel de audio, en lugar de estar en una escala lineal de cero a uno se han implementado en una escala cuadrática de cero a uno, de forma que el rango de valores es el mismo, pero al comienzo de la barra los cambios apenas son perceptibles de cara a poder ajustar lo máximo posible el nivel de sonido, mientras que si se quiere un nivel alto de sonido se pueda conseguir al mantener escala.

\subsection{Pantalla}

Para las opciones de pantalla, debido a distintas trabas impuestas por el framework utilizado, solo disponemos de dos opciones, activar o desactivar la sincronización vertical, y mostrar o no mostrar un contador de los frames generados por segundo:

\begin{figure}[H]
	\centering
	\includegraphics[width=0.6\textwidth]{"opciones/pantalla.png"}
	\caption{Opciones de pantalla en Lose To Win.}\label{figure:pantalla}
\end{figure}

Estas dos opciones, debido a que es necesario recargar la ventana completa, son las únicas opciones que se harán efectivas una vez guardemos los cambios y volvamos al menú principal.

Como comentaremos más adelante, una de las opciones pendientes más claras es la posibilidad de cambiar el tamaño de pantalla, que ha sido imposible de implementar por la poca flexibilidad de Love2D, aunque en un futuro cercano es uno de nuestros objetivos a añadir.

\subsection{Controles}

Los controles de Lose To Win son muy simples, de ahí que las opciones para modificarlos sean muy simples. Simplementen se tratan de tres botones que nos permitirán actualizar las tres teclas utilizadas, saltar, moverse a la izquierda y moverse a la derecha:

\begin{figure}[H]
	\centering
	\includegraphics[width=0.6\textwidth]{"opciones/controles_especifico.png"}
	\caption{Opciones de controles en Lose To Win.}\label{figure:controles_especifico}
\end{figure}

Esta sección de la configuración se basa en dos botones para cambiar el jugador al que modificaremos los controles, y los tres botones para cada acción.

Además, en caso de que una tecla ya esté en uso, o no sea válida, como la tecla de escape por ejemplo, mostrará un error:

\begin{figure}[H]
    \centering
	 \begin{subfigure}[b]{0.49\textwidth}
		 \centering
		 \includegraphics[width=\textwidth]{"opciones/tecla_asignada_especifico.png"}
		 \caption{Error al seleccionar una tecla ya asignada.}\label{fig:tecla_asignada_especifico}
	 \end{subfigure}
	 \begin{subfigure}[b]{0.49\textwidth}
		 \centering
		\includegraphics[width=\textwidth]{"opciones/tecla_invalida_especifico.png"}
		\caption{Error al seleccionar una tecla inválida.}\label{fig:tecla_invalida_especifico}
   \end{subfigure}
	\caption{Posibles errores al seleccionar los controles.}\label{fig:errores_controles}
\end{figure}

Por último, comentar que otra de las mejoras que tenemos planteada añadir en un futuro es la posibilidad de utilizar mandos como método de entrada.

\section{Modos de juego}

Actualmente, como hemos visto en el menú de juego, tenemos dos modos de juego, un modo de prácticas y otro el modo de juego normal.

\subsection{Modo de práctica}

En este modo podremos prácticas los distintos minijuegos disponibles, de cara a aprender mecánicas, acostumbrarte a los controles y ver como funcionan dichos minijuegos antes de jugar en la mayor competición que podrás disfrutar en un juego de tal calibre, llegando a enfrentarse directamente con los mejores de su género.

En este modo el juego podrá ser interrumpido en cualquier momento pulsando la tecla escape, permitiendo a los jugadores practicar todo lo que consideren necesario.

\subsubsection{Minijuegos disponibles}

Actualmente disponemos de tres minijuegos:

\begin{enumerate}
	\item Bomb tag.
	\item Death ball.
	\item Virus fall.
\end{enumerate}

En el primer minijuego el objetivo será retener la bomba hasta que esta explote, haciéndote volar por los aires y así poder lograr tu objetivo de ser un perdedor. Pero no será tan simple como esperarte sentado a que dicha bomba explote, tus adversarios no cesarán su persecución hasta arrebatarte tu preciada y deseada muerte:

\begin{figure}[H]
	\centering
	\includegraphics[width=\textwidth]{"modos/juego_bomba.png"}
	\caption{Cuatro jugadores jugando a Bomb Tag.}\label{figure:juego_bomba}
\end{figure}


En el segundo minijuego ha aparecido una extraña bola chisporroteante que rebotará por el escenario, reduciendo a la nada a cualquier afortunado jugador viscoso que se cruce en su camino, concediendole su ansiada derrota:

\begin{figure}[H]
	\centering
	\includegraphics[width=\textwidth]{"modos/juego_bola.png"}
	\caption{Cuatro jugadores jugando a Death Ball.}\label{figure:juego_bola}
\end{figure}

Por último, una plaga de virus mortales para los temibles (y cucos) personajes caerán del cielo, que no dudarán en infectar y acabar con la vida de todos y cada uno de los jugadores. A diferencia de en la vida real, gana quién se infecte antes por no mantenerse alejado del peligroso virus.

\begin{figure}[H]
	\centering
	\includegraphics[width=\textwidth]{"modos/juego_virus.png"}
	\caption{Cuatro jugadores jugando a Virus Fall.}\label{figure:juego_virus}
\end{figure}

\subsection{Modo de juego}

En el modo principal de juego los jugadores jugarán a los tres minijuegos durante cierto tiempo preestablecido, separando cada minijuego por una animación de un telón. En este modo, en lugar de mostrar la tecla para parar el minijuego, se mostrará el tiempo restante del minijuego y no será posible parar este modo sin que finalicen todos los minijuegos:

\begin{figure}[H]
	\centering
	\includegraphics[width=\textwidth]{"modos/virus_play_state.png"}
	\caption{Cuatro jugadores jugando a Virus Fall.}\label{figure:virus_play_state}
\end{figure}

\begin{figure}[H]
	\centering
	\includegraphics[width=\textwidth]{"modos/cortinas_cambio_minijuego.png"}
	\caption{Cortinas mostrando el cambio de un minijuego.}\label{figure:cortinas_cambio_minijuego}
\end{figure}

Como detalle dejar claro que el minijuego se para por completo en el momento que comienzan a moverse las cortinas, evitando que ocurran muertes fuera de plano.

\section{Recursos multimedia}

Todo el contenido multimedia utilizado en Lose To Win utiliza una licencia libre Creative Commons, diferenciando entre CC-0 y CC-BY. Aunque actualmente el juego se encuentre en un estado pre-lanzamiento, tenemos como objetivo ampliar la experiencia de usuario mejorando el contenido multimedia actual, incluyendo nuevo mapeado, variedad de apariencias para los personajes, etc.

\subsection{Arte}

\subsubsection{Sprites de los jugadores}

El diseño de los personajes consta de un slime que cuenta con cinco animaciones distintas, de las cuales se han utilizado cuatro de ellas (en reposo, andando, saltando y muriendo). Además, el resto de personajes consta en variaciones cromáticas del mismo, siendo sencillo plantear la posibilidad de ampliar el repertorio de diseños de los jugadores.

\begin{figure}[H]
	\centering
	\includegraphics[width=\textwidth]{"multimedia/blue_slime_atlas.png"}
	\caption{Altas de animación de uno de los jugadores.}\label{figure:blue_slime_atlas}
\end{figure}

Por otra parte, al solo necesitar cuatro animaciones distintas, facilitamos enormente la introducción de aspectos completamente nuevos, similares al ejemplo mostrado a continuación, que se descartaron finalmente debido a necesitar un rediseño de algunas de sus animaciones.

\begin{figure}[H]
	\centering
	\includegraphics[width=\textwidth]{"multimedia/dinos.png"}
	\caption{Altas de animación descartado.}\label{figure:dinos}
\end{figure}


\subsubsection{Sprites de los mapas}

Al igual que con en el diseño de los personajes, se ha optado por un diseño simple y suficientemente variado, contando con plataformas detalladas y amigables a la vista. Los sprites utilizados incluyen baldosas que conformarán todo el escenario, paredes, techos, plataformas, etc.

\begin{figure}[H]
	\centering
	\includegraphics[width=\textwidth]{"multimedia/terrain.png"}
	\caption{Altas con los distintos elementos de las plataformas.}\label{figure:terrain}
\end{figure}

Además, se ha utilizado una imagen de fondo que se integra con el estilo elegido e incluso forma parte del fondo utilizado tanto en los escenarios como en la interfaz del juego.

\begin{figure}[H]
	\centering
	\includegraphics[width=\textwidth]{"multimedia/landscape.png"}
	\caption{Paisaje de fondo.}\label{figure:landscape}
\end{figure}

\subsubsection{Sprites de los objetos utilizados}

Por último se han utilizado diversos sprites simples pero distintivos para tanto representar los iconos informativos usados en las plataformas de los distintos menús como de los objetos y amenazas que forman parte de cada uno de los minijuegos. Como ejemplo, incluimos el sprite de la bola de energía del minijuego Death Ball, que cuenta con una simple animación de dos frames.

\begin{figure}[H]
	\centering
	\includegraphics[width=0.3\textwidth]{"multimedia/energy_ball.png"}
	\caption{Sprites de animación de la bola de energía.}\label{figure:energy_ball}
\end{figure}

\subsection{Mapeado}

A la hora de crear los mapas utilizados hemos decidido utilizar la aplicación \href{https://www.mapeditor.org/}{Tiled}, ya que su uso es muy sencillo y contamos con una biblioteca de Lua que los integra en Love2D, como comentaremos más adelante.

\subsubsection{Creación de los mapas y menús: Tiled}

Para crear los mapas simplemente hemos creado un mapa en Tiled del tamaño deseado y utilizado para la pantalla en nuestro juego, con cuadrículas de 32 píxeles para ajustar todo, y hemos añadido los dibujos necesarios del terreno. Gracias a la biblioteca \href{https://github.com/karai17/Simple-Tiled-Implementation}{STI} es posible añadir a los objetos añadidos en Tiled colisiones y categorías para gestionar las físicas con estos objetos desde el código.

\begin{figure}[H]
	\centering
	\includegraphics[width=\textwidth]{"mapa_menu_principal.png"}
	\caption{Mapa del menú principal en Tiled.}\label{figure:mapa_menu_principal}
\end{figure}

También se ha podido añadir información que se utilizará en el juego, como los puntos de aparición de los jugadores, entre otros.

El resto de mapas ha sido creado de la misma forma, añadiendo plataformas, sensores, y demás donde era necesario.


\subsection{Sonido}

\subsubsection{Música}

En el apartado de banda sonora, se ha optado por una única canción simple y corta en bucle pero que a su vez es agradable y adictiva. En un futuro, se prevee ampliar el repertorio de banda sonora, contando con diversos tipos de canciones que se adecueen al flujo del juego, contando con canciones más tranquilas para los menús y más frenéticas para los minijuegos.

\subsubsection{Efectos de sonido}

En el apartado de SFX o efectos de sonido el trabajo se ha centrado principalmente en dotar a los personajes de efectos sonoros que mejorasen la experiencia del usuario, aportando retroalimentación a los jugadores en función de la acción realizada, ya fuese saltar, aterrizar en una plataforma o conseguir morirse.

También se ha incluido un efecto sonoro como prueba al minijuego Death Ball que se activa al generar cualquier colisión de la bola de energía, ya fuese con los jugadores o el esceario. Para el trabajo futuro, se va a trabajar en seguir mejorando esta experiencia de usuario aportando mayor retroalimentación a los jugadores mediante efectos sonoros.

\section*{Desarrollo}
\label{sec:desarrollo}
\addcontentsline{toc}{section}{\nameref{sec:desarrollo}}

En esta sección comentaremos los aspectos y tomas de decisiones más importantes en el proceso del desarrollo del proyecto.

\section{Lenguaje y framework utilizados}

En un principio se decidió utilizar el lenguaje de programación Lua con el framework Love2D ya que parecía un entorno sencillo, amigable, y rápido de utilizar. Sin embargo, y aunque eramos conscientes de haber escogido un entorno de más bajo nivel, esto fue un error por los siguientes motivos:

\begin{enumerate}
	\item Falta de estructuras de datos al uso: Lua solo dispone de tablas como estructuras de datos, además de no disponer de una estructura orientada a objetos, permitiendo clases complejas personalizadas.
	\item Documentación anticuada: Lua se trata de un lenguaje muy permisivo, por lo que toda la información que encontrada se trata de parches que no se deberían utilizar, funciones sin soporte, o se había modificado su uso versiones más actuales pero no se había actualizado la documentación. Este problema está también presente en la documentación de Love2D, aunque en menor medida.
	\item Falta de herramientas por parte de Love2D: Aunque Love2D aporta una base para el desarrollo de un videojuego, tras intentar realizar un proyecto de relativa (aunque pequeña) envergadura que vaya más allá de una pantalla mostrando cierta información, es claramente visible la falta de herramientas que o bien no existían, o estaban incompletas, llevando a una situación en la que gran parte de tareas que son comunes en otros entornos como Godot, Unity o Unreal Engine son triviales, en Love2D ha sido necesario implementarlas desde cero, sumando los contratiempos provistos por la falta de estructuras, funcionalidades y herramientas de Lua.
\end{enumerate}

Por estos motivos, aunque hemos aprendido en gran profundidad las etapas, herramientas y pasos necesarios para desarrollar un videojuego y hemos logrado crear un proyecto que bajo nuestro punto de vista es de gran calidad, tenemos planeado reescribirlo en otro entorno más completo, aunque nos gustaría mantener el desarrollo a bajo nivel.

\section{Bibliotecas externas utilizadas}

Hemos utilizado las siguientes bibliotecas:

\begin{itemize}
	\item \href{https://github.com/Dynodzzo/Lua_INI_Parser}{LIP}: Biblioteca para cargar archivos INI.
	\item \href{https://github.com/kikito/middleclass}{middleclass}: Biblioteca para usar clases en Lua.
	\item \href{https://github.com/vrld/suit}{SUIT}: Biblioteca para gestionar menús con sliders, checkbox, etc.
	\item \href{https://github.com/karai17/Simple-Tiled-Implementation}{STI}: Biblioteca para cargar mapas de Tiled en Love2D.
\end{itemize}

Estas bibliotecas nos han permitido suplir muchos de los problemas de Lua, aunque eso no ha evitado el tiempo y trabajo perdido hasta llegar a una iteración en la que hemos llegado a implementar todo el proyecto utilizando estas herramientas.

\section{Estructura del proyecto}

El proyecto se estructura en un fichero principal, \texttt{main.lua}, que será el fichero gestionado por Love2D debido a las necesidades de este framework. Este fichero delegará todas las funciones de Love2D (entrada por teclaro, ratón, actualización, dibujado, eventos, etc del juego) a nuestra propia clase \texttt{Game}.

\subsection{Clase gestora Game}

Nuestra clase \texttt{Game} será la encargada de gestionar en todo momento el estado del juego, que mapa se está cargando, que menú se debe mostrar, cuantos jugadores activos existen, así como realizar todos los cambios necesarios para mantener la actualización, dibujado y otras tareas del juego. Esto no implica que esta clase, a su vez, delegue el trabajo concreto a las distintas clases que utilizaremos, de forma que funcione como un gestor del juego.

\subsection{Clase genérica GameObject}

Esta clase será una clase interfaz que servirá como plantilla para todos los objetos que encontremos en el juego. Estos objetos son los jugadores, con la clase \texttt{Player} y los distintos objetos necesarios para los minijuegos, como la bola de energía, o los virus.

Estas clases gestionarán individualmente su comportamiento, como se deben mover, que tienen permitido, que formas y cuerpos tendrán, sus animaciones, entre otros.

Con estas clases hemos conseguido que todo lo que recaiga de la gestión de los jugadores y otros objetos con respecto a su creación, dibujado, actualización, etc, se realice en su respectiva sección sin que intervengan otras variables y casos generales del juego.

\subsection{Clase Level}

Esta clase será la clase que contendrá el mapa y los objetos que formarán parte de ese mapa, como los jugadores, objetos, entre otros.

Nos servirá para centralizar y gestionar los distintos objetos activos.

\subsection{Clase genérica MiniGame}

Esta clase se tratará de una plantilla para los minijuegos. Esta clase definirá el comportamiento común de los minijuegos, permitiendo una mayor escalabilidad, haciendo que el implementar un minijuego sea simplemente gestionar las características específicas de cada minijuego, haciendo más fácil y viable el permitir minijuegos personalizados.

Los minijuegos implementados actualmente son clases derivadas de esta, implementando la lógica de cada minijuego en concreto.

Otro uso de esta clase ha sido las pantallas de menús. Debido a que en nuestro juego los menús forman parte de la jugabilidad del proyecto, los propios menús son de tipo \texttt{MiniGame} ya que necesitan gestionar su propia lógica de a que parte pueden saltar, que sensores son necesarios, entre otros.


\subsection{Otras clases auxiliares}

También disponemos de otras clases auxiliares de cara a automatizar y gestionar tareas concretas, como la clase \texttt{SoundManager} que se encargará de gestionar todos los sonidos del minijuego, la clase \texttt{InterfaceBox} que se encargará de la interfaz inferior del juego o la clase \texttt{TextBox} que surgió de la necesidad de tener cajas de texto para mostrar información por pantalla en momento puntuales.


\section*{Trabajo futuro}
\label{sec:futuro}
\addcontentsline{toc}{section}{\nameref{sec:futuro}}

De cara a seguir con el desarrollo del proyecto, nos gustaría realizar las siguientes mejoras y ampliaciones de funcionalidad, previo estudio de la posibilidad de reimplementar este proyecto en un entorno distinto:

\begin{itemize}
	\item Mejorar el sistema de puntuaciones.
	\item Selector de mapas en el modo en práctica.
	\item Permitir a los usuarios añadir mapas personalizados.
	\item Permitir a los usuarios añadir minijuegos personalizados.
	\item Añadir soporte de mandos como dispositivo de entrada.
	\item Multijugador en red.
	\item Modo un jugador (para algunos minijuegos).
	\item Mayor versatilidad en la configuración de minijuegos.
\end{itemize}


\section*{Manual de usuario}
\label{sec:manual}
\addcontentsline{toc}{section}{\nameref{sec:manual}}


\section{Como descargar y ejecutar el videojuego}

\subsection{Ejecutando el código fuente manualmente con Love2D}

\subsection{Utilizando la versión de lanzamiento}

\section{Como jugar al videojuego}


\end{document}
