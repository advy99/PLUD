\documentclass[12pt, spanish]{article}
\usepackage[spanish]{babel}
\selectlanguage{spanish}
%\usepackage{natbib}
\usepackage{url}
\usepackage[utf8]{inputenc}
\usepackage{graphicx}
\graphicspath{{images/}}
\usepackage{parskip}
\usepackage{fancyhdr}
\usepackage{vmargin}
\usepackage{multirow}
\usepackage{float}
\usepackage{chngpage}

\usepackage{amsfonts}

\usepackage{caption}
\usepackage{subcaption}

\usepackage{hyperref}
\usepackage[
    type={CC},
    modifier={by-nc-sa},
    version={4.0},
]{doclicense}

\hypersetup{
    colorlinks=true,
    linkcolor=blue,
    filecolor=magenta,
    urlcolor=cyan,
}

% para codigo
\usepackage{listings}
\usepackage{xcolor}



%% configuración de listings

\definecolor{listing-background}{HTML}{F7F7F7}
\definecolor{listing-rule}{HTML}{B3B2B3}
\definecolor{listing-numbers}{HTML}{B3B2B3}
\definecolor{listing-text-color}{HTML}{000000}
\definecolor{listing-keyword}{HTML}{435489}
\definecolor{listing-identifier}{HTML}{435489}
\definecolor{listing-string}{HTML}{00999A}
\definecolor{listing-comment}{HTML}{8E8E8E}
\definecolor{listing-javadoc-comment}{HTML}{006CA9}

\lstdefinestyle{eisvogel_listing_style}{
  language         = python,
%$if(listings-disable-line-numbers)$
%  xleftmargin      = 0.6em,
%  framexleftmargin = 0.4em,
%$else$
  numbers          = left,
  xleftmargin      = 0em,
 framexleftmargin = 0em,
%$endif$
  backgroundcolor  = \color{listing-background},
  basicstyle       = \color{listing-text-color}\small\ttfamily{}\linespread{1.15}, % print whole listing small
  breaklines       = true,
  frame            = single,
  framesep         = 0.19em,
  rulecolor        = \color{listing-rule},
  frameround       = ffff,
  tabsize          = 4,
  numberstyle      = \color{listing-numbers},
  aboveskip        = 1.0em,
  belowskip        = 0.1em,
  abovecaptionskip = 0em,
  belowcaptionskip = 1.0em,
  keywordstyle     = \color{listing-keyword}\bfseries,
  classoffset      = 0,
  sensitive        = true,
  identifierstyle  = \color{listing-identifier},
  commentstyle     = \color{listing-comment},
  morecomment      = [s][\color{listing-javadoc-comment}]{/**}{*/},
  stringstyle      = \color{listing-string},
  showstringspaces = false,
  escapeinside     = {/*@}{@*/}, % Allow LaTeX inside these special comments
  literate         =
  {á}{{\'a}}1 {é}{{\'e}}1 {í}{{\'i}}1 {ó}{{\'o}}1 {ú}{{\'u}}1
  {Á}{{\'A}}1 {É}{{\'E}}1 {Í}{{\'I}}1 {Ó}{{\'O}}1 {Ú}{{\'U}}1
  {à}{{\`a}}1 {è}{{\'e}}1 {ì}{{\`i}}1 {ò}{{\`o}}1 {ù}{{\`u}}1
  {À}{{\`A}}1 {È}{{\'E}}1 {Ì}{{\`I}}1 {Ò}{{\`O}}1 {Ù}{{\`U}}1
  {ä}{{\"a}}1 {ë}{{\"e}}1 {ï}{{\"i}}1 {ö}{{\"o}}1 {ü}{{\"u}}1
  {Ä}{{\"A}}1 {Ë}{{\"E}}1 {Ï}{{\"I}}1 {Ö}{{\"O}}1 {Ü}{{\"U}}1
  {â}{{\^a}}1 {ê}{{\^e}}1 {î}{{\^i}}1 {ô}{{\^o}}1 {û}{{\^u}}1
  {Â}{{\^A}}1 {Ê}{{\^E}}1 {Î}{{\^I}}1 {Ô}{{\^O}}1 {Û}{{\^U}}1
  {œ}{{\oe}}1 {Œ}{{\OE}}1 {æ}{{\ae}}1 {Æ}{{\AE}}1 {ß}{{\ss}}1
  {ç}{{\c c}}1 {Ç}{{\c C}}1 {ø}{{\o}}1 {å}{{\r a}}1 {Å}{{\r A}}1
  {€}{{\EUR}}1 {£}{{\pounds}}1 {«}{{\guillemotleft}}1
  {»}{{\guillemotright}}1 {ñ}{{\~n}}1 {Ñ}{{\~N}}1 {¿}{{?`}}1
  {…}{{\ldots}}1 {≥}{{>=}}1 {≤}{{<=}}1 {„}{{\glqq}}1 {“}{{\grqq}}1
  {”}{{''}}1
}
\lstset{style=eisvogel_listing_style}


\usepackage[default]{sourcesanspro}

\setmarginsrb{2 cm}{1 cm}{2 cm}{2 cm}{1 cm}{1.5 cm}{1 cm}{1.5 cm}

\title{Entrega final:\\
Estado actual del proyecto.\hspace{0.05cm} }
\author{Guillermo Sandoval Schmidt\\ Antonio David Villegas Yeguas}
\date{\today}

\makeatletter
\let\thetitle\@title
\let\theauthor\@author
\let\thedate\@date
\makeatother

\pagestyle{fancy}
\fancyhf{}
\rhead{\theauthor}
\lhead{\thetitle}
\cfoot{\thepage}

\begin{document}

%%%%%%%%%%%%%%%%%%%%%%%%%%%%%%%%%%%%%%%%%%%%%%%%%%%%%%%%%%%%%%%%%%%%%%%%%%%%%%%%%%%%%%%%%

\begin{titlepage}
    \centering
    \vspace*{0.3 cm}
    \includegraphics[scale = 0.50]{ugr.png}\\[0.7 cm]
    %\textsc{\LARGE Universidad de Granada}\\[2.0 cm]
    \textsc{\large 4º CSI 2020/21}\\[0.5 cm]
    \textsc{\large Grado en Ingeniería Informática}\\[0.5 cm]
    \rule{\linewidth}{0.2 mm} \\[0.2 cm]
    { \huge \bfseries \thetitle}\\
    \rule{\linewidth}{0.2 mm} \\[1 cm]

    \begin{minipage}{0.4\textwidth}
        \begin{flushleft} \large
            \emph{Autores:}\\
            \theauthor\\
            \end{flushleft}
            \end{minipage}~
            \begin{minipage}{0.4\textwidth}
            \begin{flushright} \large
            \emph{Asignatura: \\
            Programación Lúdica}   \\
        \end{flushright}
    \end{minipage}\\[0.5cm]

    {\large \thedate}\\[0.5cm]
    %{\url{https://github.com/advy99/AA/}}
    {\doclicenseThis}

    \vfill

\end{titlepage}

%%%%%%%%%%%%%%%%%%%%%%%%%%%%%%%%%%%%%%%%%%%%%%%%%%%%%%%%%%%%%%%%%%%%%%%%%%%%%%%%%%%%%%%%%

\tableofcontents
\pagebreak

%%%%%%%%%%%%%%%%%%%%%%%%%%%%%%%%%%%%%%%%%%%%%%%%%%%%%%%%%%%%%%%%%%%%%%%%%%%%%%%%%%%%%%%%%

\section{Integrantes del grupo}

Los integrantes del grupo somos:

\begin{itemize}
	\item Guillermo Sandoval Schmidt
	\item Antonio David Villegas Yeguas
\end{itemize}

\section*{Estudio previo}
\label{sec:estudio_previo}
\addcontentsline{toc}{section}{\nameref{sec:estudio_previo}}

\section{Propuesta}

\subsection{Título provisional}

Como título provisional hemos decidido \textbf{Lose to Win}, debido a la principal mecánica del videojuego que explicaremos en la descripción.

\subsection{Descripción del juego, género y público objetivo}


Lose to Win es un juego multijugador de partidas rápidas basado en minijuegos, donde los jugadores se enfrentan todos contra todos o por equipos para alzarse con la victoria.

Cuenta con una mecánica transversal a los diversos minijuegos, ya que para ganar debes perder el minijuego.

Algunos de los minijuegos que hemos pensado son:

\begin{enumerate}
	\item Bomb tag: uno de los jugadores aparece con una bomba, al tocar al jugador con la bomba se la robas. Gana el jugador que tenga la bomba cuando explote.
	\item Egg Giver: el clásico juego de coger huevos y llevarlos a tu base pero al revés, gana el jugador que menos huevos tenga en su base.
	\item Spike Ball: similar al minijuego de “Move or Die” de mismo nombre, pero el objetivo es que la bola de pinchos te mate, ganando el jugador que más veces muera.
	\item Generous Guy: recoge objetos del color del rival para que gane puntos y ganar siendo el jugador con menos puntos.
\end{enumerate}


Nuestra idea no se encaja en un único género, debido a la gran variedad de minijuegos y objetivos de las distintas partidas, aun así, los principales géneros de nuestra idea serían acción, party game y plataformas.

En principio el público objetivo sería cualquier persona, independientemente de la edad (PEGI 3), aunque es cierto que esto variará dependiendo del estilo gráfico del juego y como de explícito sea con respecto a la violencia de los minijuegos, tanto a nivel de que tipos de minijuegos como a nivel artístico, lo que conllevaría un PEGI 12.

\section{Estudio de mercado}

Aunque hemos encontrado una gran cantidad de videojuegos similares a nuestra idea, finalmente hemos escogido estos tres ya que son los más parecidos a nuestra propuesta.

\subsection{Move or Die}

Se trata de un videojuego para cuatro jugadores en el que cada jugador controla un personaje cuya vida se reduce rápidamente si el jugador deja de moverse por un momento, se regenerá si reanuda el movimiento. Cada ronda contará con distintas reglas o modificadores, que varían entre los distintos modos. El desafío surge cuando los jugadores siguen moviéndose para ganar, mientras evitan peligros como picos o bloques que caen. Los jugadores también pueden empujarse entre sí.


\subsubsection{Detalles sobre el videojuego}

\begin{itemize}
	\item \textbf{Compañia}: Those Awesome Guys
	\item \textbf{Plataformas}: Windows, Linux, macOS, PlayStation 4
	\item \textbf{Modelo de negocio}: Juego de pago (Pay-to-play)
	\item \textbf{Web}: \url{http://www.moveordiegame.com}
\end{itemize}

\subsubsection{Capturas del videojuego}

\begin{figure}[H]
  \centering
   \includegraphics[width=\textwidth]{"competencia/mod_juego.jpg"}
	\caption{Imagen de Move or Die dentro de una partida.}
\end{figure}

\begin{figure}[H]
  \centering
   \includegraphics[width=\textwidth]{"competencia/mod_editor.jpg"}
	\caption{Editor de niveles de Move or Die.}
\end{figure}

\begin{figure}[H]
  \centering
   \includegraphics[width=\textwidth]{"competencia/mod_lobby.jpg"}
	\caption{Lobby de Move or Die.}
\end{figure}

\subsubsection{Aspectos positivos}

\begin{itemize}
	\item Juego local y online.
	\item De 1 a 4 jugadores.
	\item Incluye editor de niveles.
	\item Facilidad para crear mods.
	\item Compatible con mando y teclado.
	\item Juego multiplataforma (puedes jugar con una persona que no esté jugando en tu misma plataforma).
	\item Variedad de minijuegos.
\end{itemize}

\subsubsection{Aspectos negativos}

\begin{itemize}
	\item Requiere muchas horas desbloquear los distintos modos de juego.
	\item Modos de juego bloqueados que solo se pueden desbloquear con “monedas del juego”.
	\item Malos servidores para jugar online.
\end{itemize}

\subsection{Party Panic}

Es un juego multijugador de hasta cuatro jugadores en el que compiten unos contra otros en distintos minijuegos. Gana el jugador que más minijuegos complete victorioso.

\subsubsection{Detalles sobre el videojuego}

\begin{itemize}
	\item \textbf{Compañia}: Everglow Interactive
	\item \textbf{Plataformas}: Windows, macOS, Linux, XBOX ONE, PlayStation 4
	\item \textbf{Modelo de negocio}: Juego de pago (Pay-to-play)
	\item \textbf{Web}: \url{https://partypanicgame.com/}
\end{itemize}

\subsubsection{Capturas del videojuego}


\begin{figure}[H]
  \centering
   \includegraphics[width=\textwidth]{"competencia/pp_juego.jpg"}
	\caption{Imagen de Party Panic dentro de una partida.}
\end{figure}

\begin{figure}[H]
  \centering
   \includegraphics[width=\textwidth]{"competencia/pp_juego2.jpg"}
	\caption{Imagen de Party Panic dentro de una partida.}
\end{figure}


\begin{figure}[H]
  \centering
   \includegraphics[width=\textwidth]{"competencia/pp_lobby.jpg"}
	\caption{Lobby de Party Panic.}
\end{figure}

\subsubsection{Aspectos positivos}

\begin{itemize}
	\item Juego local y online.
	\item De 1 a 4 jugadores.
	\item Compatible con mando y teclado.
\end{itemize}

\subsubsection{Aspectos negativos}

\begin{itemize}
	\item No tiene editor de niveles ni soporte de mods.
	\item La IA no funciona correctamente.
	\item Interfaz muy básica y poco accesible.
	\item En una partida no se permiten jugadores de distintas plataformas.
	\item Falta variedad de jugadores.
	\item Malos servidores.
\end{itemize}

\subsection{Duck Game}

Es un videojuego de acción donde los personajes son patos que incluye características de videojuegos de disparos y plataformas. El juego tiene un modo de un jugador al igual que un modo multijugador en línea con un límite de hasta siete jugadores más. En este modo de juego el jugador que reciba un solo disparo muere y el jugador sobreviviente gana la ronda.


\subsubsection{Detalles sobre el videojuego}

\begin{itemize}
	\item \textbf{Compañia}: Adult Swim Games
	\item \textbf{Plataformas}: Windows, PlayStation 4, Switch
	\item \textbf{Modelo de negocio}: Juego de pago (Pay-to-play)
	\item \textbf{Web}: \url{https://www.adultswim.com/games/duck-game}
\end{itemize}

\subsubsection{Capturas del videojuego}

\begin{figure}[H]
  \centering
   \includegraphics[width=\textwidth]{"competencia/dg_juego.jpg"}
	\caption{Imagen de una partida de Duck Game.}
\end{figure}

\begin{figure}[H]
  \centering
   \includegraphics[width=\textwidth]{"competencia/dg_muestra.jpg"}
	\caption{Imagen de una escena de Duck Game.}
\end{figure}


\begin{figure}[H]
  \centering
   \includegraphics[width=\textwidth]{"competencia/dg_lobby.jpg"}
	\caption{Lobby de Duck Game.}
\end{figure}

\subsubsection{Aspectos positivos}

\begin{itemize}
	\item Juego local y online.
	\item De 2 a 8 jugadores.
	\item Incluye editor de niveles.
	\item Compatible con mando y teclado.
	\item En una partida no se permiten jugadores de distintas plataformas.
\end{itemize}

\subsubsection{Aspectos negativos}

\begin{itemize}
	\item Malos servidores.
	\item No permite modificar el esquema de control.
	\item No permite utilizar múltiples mandos en juego local.
	\item No permite permite varios jugadores
\end{itemize}

\section*{Estado actual}
\label{sec:estado_actual}
\addcontentsline{toc}{section}{\nameref{sec:estado_actual}}

\section{Resumen del estado actual}

Actualmente el proyecto se encuentra en un estado previo al lanzamiento, donde todas las ideas que planteamos están finalizadas, pero es necesario realizar unas pruebas en profundidad de cara a calibrar y mejorar la experiencia final de usuario.

Presentamos un juego multijugador competitivo, que actualmente se ejecuta localmente pero con la idea de llevarlo a un juego en red. Lose To Win cuenta con unos controles e interfaces simples e intuitivas de cara a que cualquier persona con solo mirar la pantalla sea capaz de jugar, teniendo una curva de aprendizaje nula y permitiendo saber jugar y divertirse desde el primer momento.

En Lose To Win jugaras partidas rápidas y frenéticas, compuestas por diversos minijuegos muy simples, donde el objetivo será perder en el minijuego, invirtiendo la forma tradicional de entender los juegos.

Actualmente contamos con tres minijuegos completados, pudiendo jugar en modo práctica o bien jugando a todos los minijuegos con el objetivos de llegar a lo alto de la tabla de puntuaciones globales. Con respecto a las opciones, contamos con diversas opciones para personalizar el idioma, sonido, pantalla y controles. En todas estas secciones entraremos en detalle más adelante.


\begin{figure}[H]
  \centering
   \includegraphics[width=\textwidth]{"menu_principal.png"}
	\caption{Menú principal de Lose To Win, con cuatro jugadores.}\label{figure:titulo}
\end{figure}


\section{Interfaz de usuario}

La interfaz de usuario se trata de una interfaz simple, dividida en dos secciones, la pantalla de juego, y una pantalla donde los usuarios podrán ver tanto sus controles y como añadir jugadores si se encuentran en un menú o su puntuación si se encuentran jugando:

\begin{figure}[H]
  \centering
	\includegraphics[width=\textwidth]{"interfaz/interfaz_inferior_anadir.png"}
  \caption{Interfaz inferior con dos jugadores en un menú.}\label{figure:interfaz_inferior_anadir}
\end{figure}

\begin{figure}[H]
  \centering
	\includegraphics[width=\textwidth]{"interfaz/interfaz_inferior_jugando.png"}
  \caption{Interfaz inferior con cuatro jugadores jugando.}\label{figure:interfaz_inferior_jugando}
\end{figure}

Con respecto al resto de interfaz, esta se encuentra totalmente integrada en el propio juego, los jugadores con sus personajes se moverán por plataformas que les llevarán a las distintas secciones disponibles, de forma que la adaptación a los controles del juego se hace de una forma escalada, en la que los jugadores pueden aprender a manejar a los personajes desde el principio, navegando por la interfaz. La única excepción será en el menú de opciones, donde la modificación de opciones se realizará con el ratón por motivos de comodidad frente al gran número de opciones disponibles.

De cara a poder navegar por el juego y sus menús será necesario que los jugadores se pongan de acuerdo sobre a que sección han de navegar, ya que el juego pedirá que todos los jugadores activos se encuentren en la misma plataforma para pasar de una sección a otra. Todas las plataformas que llevan a secciones están marcadas con símbolos en el mapa, además de tener un texto de acompañamiento junto a una cuenta atrás para asegurar que se trata del menú correcto:

\begin{figure}[H]
  \centering
	\includegraphics[width=\textwidth]{"interfaz/menu_no_funciona_num_jugadores.png"}
  \caption{El menú no se activa aunque dos jugadores quieran ir a esa sección, ya que otros dos no se han decidido.}\label{figure:menu_no_funciona_num_jugadores}
\end{figure}

\begin{figure}[H]
  \centering
	\includegraphics[width=\textwidth]{"interfaz/esperando_practicar.png"}
  \caption{Cuatro jugadores pasando a otro menú.}\label{figure:esperando_practicar}
\end{figure}

\begin{figure}[H]
  \centering
	\includegraphics[width=\textwidth]{"interfaz/menu_funciona_num_jugadores.png"}
  \caption{Cambiando el número de jugadores, seguimos necesitando a la totalidad de los jugadores activos.}\label{figure:menu_funciona_num_jugadores}
\end{figure}

Como vemos, a la vez que el jugador aprende a manejarse con su personaje, se mueve por los menús practicando el movimiento y saltos, a la vez de utilizar una interfaz simple y directa que deja claro en todo momento que es cada opción acompañándola de un mensaje.

\subsection{Secciones de la interfaz de usuario}

La interfaz de usuario está dividida de la siguiente forma:

\subsubsection{Menú principal}

Tras iniciar el juego nos encontraremos con el menú inicial, con el número mínimo de jugadores, dos:

\begin{figure}[H]
  \centering
	\includegraphics[width=\textwidth]{"interfaz/modo_inicial_defecto.png"}
  \caption{Menú inicial por defecto, con dos jugadores.}\label{figure:modo_inicial_defecto}
\end{figure}

En este menú podemos realizar cuatro opciones, de arriba a abajo y de izquierda a derecha:

\begin{enumerate}
	\item Ir al menú de juego.
	\item Ir al menú de opciones.
	\item Ir al menú de créditos.
	\item Salir del juego.
\end{enumerate}

\subsubsection{Menú de juego}

Al pasar al menú de juego, veremos lo siguiente:

\begin{figure}[H]
	\centering
	\includegraphics[width=\textwidth]{"modos/menu_juego.png"}
	\caption{Menú de juego.}\label{figure:menu_juego}
\end{figure}

Como vemos, contamos con cuatro opciones:

\begin{enumerate}
	\item Entrar a jugar.
	\item Ir al modo práctica.
	\item Ir a la tabla de puntuaciones.
	\item Volver al menú anterior.
\end{enumerate}


En la primera opción pasaremos al modo de juego por defecto, que explicaremos más adelante, en la segunda opción iremos al menú de práctica, donde podremos escoger entre los tres minijuegos disponibles actualmente y jugar de forma indefinida, con el objetivo de entender como funcionan y entrenarnos para jugar:

\begin{figure}[H]
	\centering
	\includegraphics[width=\textwidth]{"modos/menu_practica.png"}
	\caption{Menú de práctica.}\label{figure:menu_practica}
\end{figure}

Por último, en la tabla de puntuaciones podremos consultar quienes son los mejores jugadores de todo el juego, ya que como veremos más adelante en el modo de juego, tras acabar una partida el ganador podrá guardar su nombre con su puntuación en esta tabla y pasar a la historia de Lose To Win como el mayor perdedor:

\begin{figure}[H]
	\centering
	\includegraphics[width=\textwidth]{"tabla_puntuaciones.png"}
	\caption{Menú de puntuaciones.}\label{figure:tabla_puntuaciones}
\end{figure}


\subsubsection{Menú de opciones}

En el menú de opciones encontraremos una interfaz controlada por ratón para controlar las distintas configuraciones disponibles. También encontraremos dos plataformas, una a la izquierda para guardar la configuración y a la derecha para salir sin guardar los cambios realizados:

\begin{figure}[H]
	\centering
	\includegraphics[width=\textwidth]{"opciones/idiomas/spanish.png"}
	\caption{Menú de opciones.}\label{figure:menu_opciones}
\end{figure}


Con respecto a este menú y las opciones disponibles entraremos en detalle en la sección de opciones.

\subsubsection{Menú de créditos}

En este menú podremos encontrar una plataforma para volver al menú principal, además de un cuadro de texto con varias páginas que nos permitirá ver quien ha desarrollado este proyecto, así como dar crédito a todos los recursos utilizados para su desarrollo:

\begin{figure}[H]
	\centering
	\includegraphics[width=\textwidth]{"interfaz/creditos.png"}
	\caption{Menú de créditos.}\label{figure:creditos}
\end{figure}

\section{Opciones}

Con respecto a las opciones, hemos añadido distintas opciones de cara a que los usuarios puedan ajustar su modo de juego, desde el idioma, el nivel de la música y efectos de sonido, configurar opciones de la pantalla y personalizar los controles.

\begin{figure}[H]
    \centering
	 \begin{subfigure}[b]{0.49\textwidth}
		 \centering
		 \includegraphics[width=\textwidth]{"opciones/pagina1_opciones.png"}
		 \caption{Primera página del menú de opciones.}\label{fig:pagina1_opciones}
	 \end{subfigure}
	 \begin{subfigure}[b]{0.49\textwidth}
		 \centering
		\includegraphics[width=\textwidth]{"opciones/pagina2_opciones.png"}
		\caption{Segunda página del menú de opciones.}\label{fig:pagina2_opciones}
   \end{subfigure}
	\caption{Opciones disponibles en Lose To Win.}\label{fig:opciones}
\end{figure}

\subsection{Idioma}

\subsection{Sonido}

\subsection{Pantalla}

\subsection{Controles}


\section{Modos de juego}

\subsection{Modo de práctica}

\subsubsection{Minijuegos}

\subsection{Modo de juego}

\section{Tabla de puntuaciones}



\section{Recursos multimedia}

\subsection{Arte}

\subsubsection{Sprites de los jugadores}

\subsubsection{Sprites de los mapas}

\subsubsection{Sprites de los objetos utilizados}


\subsection{Mapeado}

\subsubsection{Creación de los mapas: Tiled}

\subsection{Sonido}

\subsubsection{Música}

\subsubsection{Efectos de sonido}

\section*{Desarrollo}
\label{sec:desarrollo}
\addcontentsline{toc}{section}{\nameref{sec:desarrollo}}

\section{Lenguaje y framework utilizados}

\section{Bibliotecas externas utilizadas}

\section{Estructura del proyecto}


\section*{Trabajo futuro}
\label{sec:futuro}
\addcontentsline{toc}{section}{\nameref{sec:futuro}}


\section*{Manual de usuario}
\label{sec:manual}
\addcontentsline{toc}{section}{\nameref{sec:manual}}

\section{Como descargar el videojuego}

\subsection{Ejecutando el código fuente manualmente con Love2D}

\subsection{Utilizando la versión de lanzamiento}


\section{Como ejecutar el videojuego}

\section{Como jugar al videojuego}



\end{document}
