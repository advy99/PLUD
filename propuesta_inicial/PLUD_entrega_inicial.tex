\documentclass[12pt, spanish]{article}
\usepackage[spanish]{babel}
\selectlanguage{spanish}
%\usepackage{natbib}
\usepackage{url}
\usepackage[utf8x]{inputenc}
\usepackage{graphicx}
\graphicspath{{images/}}
\usepackage{parskip}
\usepackage{fancyhdr}
\usepackage{vmargin}
\usepackage{multirow}
\usepackage{float}
\usepackage{chngpage}

\usepackage{amsfonts}

\usepackage{subcaption}

\usepackage{hyperref}
\usepackage[
    type={CC},
    modifier={by-nc-sa},
    version={4.0},
]{doclicense}

\hypersetup{
    colorlinks=true,
    linkcolor=blue,
    filecolor=magenta,
    urlcolor=cyan,
}

% para codigo
\usepackage{listings}
\usepackage{xcolor}



%% configuración de listings

\definecolor{listing-background}{HTML}{F7F7F7}
\definecolor{listing-rule}{HTML}{B3B2B3}
\definecolor{listing-numbers}{HTML}{B3B2B3}
\definecolor{listing-text-color}{HTML}{000000}
\definecolor{listing-keyword}{HTML}{435489}
\definecolor{listing-identifier}{HTML}{435489}
\definecolor{listing-string}{HTML}{00999A}
\definecolor{listing-comment}{HTML}{8E8E8E}
\definecolor{listing-javadoc-comment}{HTML}{006CA9}

\lstdefinestyle{eisvogel_listing_style}{
  language         = python,
%$if(listings-disable-line-numbers)$
%  xleftmargin      = 0.6em,
%  framexleftmargin = 0.4em,
%$else$
  numbers          = left,
  xleftmargin      = 0em,
 framexleftmargin = 0em,
%$endif$
  backgroundcolor  = \color{listing-background},
  basicstyle       = \color{listing-text-color}\small\ttfamily{}\linespread{1.15}, % print whole listing small
  breaklines       = true,
  frame            = single,
  framesep         = 0.19em,
  rulecolor        = \color{listing-rule},
  frameround       = ffff,
  tabsize          = 4,
  numberstyle      = \color{listing-numbers},
  aboveskip        = 1.0em,
  belowskip        = 0.1em,
  abovecaptionskip = 0em,
  belowcaptionskip = 1.0em,
  keywordstyle     = \color{listing-keyword}\bfseries,
  classoffset      = 0,
  sensitive        = true,
  identifierstyle  = \color{listing-identifier},
  commentstyle     = \color{listing-comment},
  morecomment      = [s][\color{listing-javadoc-comment}]{/**}{*/},
  stringstyle      = \color{listing-string},
  showstringspaces = false,
  escapeinside     = {/*@}{@*/}, % Allow LaTeX inside these special comments
  literate         =
  {á}{{\'a}}1 {é}{{\'e}}1 {í}{{\'i}}1 {ó}{{\'o}}1 {ú}{{\'u}}1
  {Á}{{\'A}}1 {É}{{\'E}}1 {Í}{{\'I}}1 {Ó}{{\'O}}1 {Ú}{{\'U}}1
  {à}{{\`a}}1 {è}{{\'e}}1 {ì}{{\`i}}1 {ò}{{\`o}}1 {ù}{{\`u}}1
  {À}{{\`A}}1 {È}{{\'E}}1 {Ì}{{\`I}}1 {Ò}{{\`O}}1 {Ù}{{\`U}}1
  {ä}{{\"a}}1 {ë}{{\"e}}1 {ï}{{\"i}}1 {ö}{{\"o}}1 {ü}{{\"u}}1
  {Ä}{{\"A}}1 {Ë}{{\"E}}1 {Ï}{{\"I}}1 {Ö}{{\"O}}1 {Ü}{{\"U}}1
  {â}{{\^a}}1 {ê}{{\^e}}1 {î}{{\^i}}1 {ô}{{\^o}}1 {û}{{\^u}}1
  {Â}{{\^A}}1 {Ê}{{\^E}}1 {Î}{{\^I}}1 {Ô}{{\^O}}1 {Û}{{\^U}}1
  {œ}{{\oe}}1 {Œ}{{\OE}}1 {æ}{{\ae}}1 {Æ}{{\AE}}1 {ß}{{\ss}}1
  {ç}{{\c c}}1 {Ç}{{\c C}}1 {ø}{{\o}}1 {å}{{\r a}}1 {Å}{{\r A}}1
  {€}{{\EUR}}1 {£}{{\pounds}}1 {«}{{\guillemotleft}}1
  {»}{{\guillemotright}}1 {ñ}{{\~n}}1 {Ñ}{{\~N}}1 {¿}{{?`}}1
  {…}{{\ldots}}1 {≥}{{>=}}1 {≤}{{<=}}1 {„}{{\glqq}}1 {“}{{\grqq}}1
  {”}{{''}}1
}
\lstset{style=eisvogel_listing_style}


\usepackage[default]{sourcesanspro}

\setmarginsrb{2 cm}{1 cm}{2 cm}{2 cm}{1 cm}{1.5 cm}{1 cm}{1.5 cm}

\title{Entrega inicial:\\
Propuesta.\hspace{0.05cm} }
\author{Guillermo Sandoval Schmidt\\ Antonio David Villegas Yeguas}
\date{\today}

\makeatletter
\let\thetitle\@title
\let\theauthor\@author
\let\thedate\@date
\makeatother

\pagestyle{fancy}
\fancyhf{}
\rhead{\theauthor}
\lhead{\thetitle}
\cfoot{\thepage}

\begin{document}

%%%%%%%%%%%%%%%%%%%%%%%%%%%%%%%%%%%%%%%%%%%%%%%%%%%%%%%%%%%%%%%%%%%%%%%%%%%%%%%%%%%%%%%%%

\begin{titlepage}
    \centering
    \vspace*{0.3 cm}
    \includegraphics[scale = 0.50]{ugr.png}\\[0.7 cm]
    %\textsc{\LARGE Universidad de Granada}\\[2.0 cm]
    \textsc{\large 4º CSI 2020/21}\\[0.5 cm]
    \textsc{\large Grado en Ingeniería Informática}\\[0.5 cm]
    \rule{\linewidth}{0.2 mm} \\[0.2 cm]
    { \huge \bfseries \thetitle}\\
    \rule{\linewidth}{0.2 mm} \\[1 cm]

    \begin{minipage}{0.4\textwidth}
        \begin{flushleft} \large
            \emph{Autores:}\\
            \theauthor\\
            \end{flushleft}
            \end{minipage}~
            \begin{minipage}{0.4\textwidth}
            \begin{flushright} \large
            \emph{Asignatura: \\
            Programación Lúdica}   \\
        \end{flushright}
    \end{minipage}\\[0.5cm]

    {\large \thedate}\\[0.5cm]
    %{\url{https://github.com/advy99/AA/}}
    {\doclicenseThis}

    \vfill

\end{titlepage}

%%%%%%%%%%%%%%%%%%%%%%%%%%%%%%%%%%%%%%%%%%%%%%%%%%%%%%%%%%%%%%%%%%%%%%%%%%%%%%%%%%%%%%%%%

\tableofcontents
\pagebreak

%%%%%%%%%%%%%%%%%%%%%%%%%%%%%%%%%%%%%%%%%%%%%%%%%%%%%%%%%%%%%%%%%%%%%%%%%%%%%%%%%%%%%%%%%

\section{Integrantes del grupo}

Los integrantes del grupo somos:

\begin{itemize}
	\item Guillermo Sandoval Schmidt
	\item Antonio David Villegas Yeguas
\end{itemize}

\section{Propuesta}

\subsection{Título provisional}

Como título provisional hemos decidido \textbf{Lose to Win}, debido a la principal mecánica del videojuego que explicaremos en la descripción.

\subsection{Descripción del juego, género y público objetivo}


Lose to Win es un juego multijugador de partidas rápidas basado en minijuegos, donde los jugadores se enfrentan todos contra todos o por equipos para alzarse con la victoria.

Cuenta con una mecánica transversal a los diversos minijuegos, ya que para ganar debes perder el minijuego.

Algunos de los minijuegos que hemos pensado son:

\begin{enumerate}
	\item Bomb tag: uno de los jugadores aparece con una bomba, al tocar al jugador con la bomba se la robas. Gana el jugador que tenga la bomba cuando explote.
	\item Egg Giver: el clásico juego de coger huevos y llevarlos a tu base pero al revés, gana el jugador que menos huevos tenga en su base.
	\item Spike Ball: similar al minijuego de “Move or Die” de mismo nombre, pero el objetivo es que la bola de pinchos te mate, ganando el jugador que más veces muera.
\end{enumerate}


Nuestra idea no se encaja en un único género, debido a la gran variedad de minijuegos y objetivos de las distintas partidas, aun así, los principales géneros de nuestra idea serían acción, party game y plataformas.

En principio el público objetivo sería cualquier persona, independientemente de la edad (PEGI 3), aunque es cierto que esto variará dependiendo del estilo gráfico del juego y como de explícito sea con respecto a la violencia de los minijuegos, tanto a nivel de que tipos de minijuegos como a nivel artístico, lo que conllevaría un PEGI 12.

\section{Estudio de mercado}

Aunque hemos encontrado una gran cantidad de videojuegos similares a nuestra idea, finalmente hemos escogido estos tres ya que son los más parecidos a nuestra propuesta.

\subsection{Move or Die}

\subsubsection{Detalles sobre el videojuego}

\begin{itemize}
	\item \textbf{Compañia}: Those Awesome Guys
	\item \textbf{Plataformas}: Windows, Linux, macOS, PlayStation 4
	\item \textbf{Modelo de negocio}: Juego de pago (Pay-to-play)
	\item \textbf{Web}: \url{http://www.moveordiegame.com}
\end{itemize}

\subsubsection{Capturas del videojuego}

\subsubsection{Aspectos positivos}

\subsubsection{Aspectos negativos}


\subsection{Party Panic}

\subsubsection{Detalles sobre el videojuego}

\begin{itemize}
	\item \textbf{Compañia}: Everglow Interactive
	\item \textbf{Plataformas}: Windows, macOS, Linux, XBOX ONE, PlayStation 4
	\item \textbf{Modelo de negocio}: Juego de pago (Pay-to-play)
	\item \textbf{Web}: \url{https://partypanicgame.com/}
\end{itemize}

\subsubsection{Capturas del videojuego}

\subsubsection{Aspectos positivos}

\subsubsection{Aspectos negativos}

\subsection{Duck Game}

\subsubsection{Detalles sobre el videojuego}

\begin{itemize}
	\item \textbf{Compañia}: Adult Swim Games
	\item \textbf{Plataformas}: Windows, PlayStation 4, Switch
	\item \textbf{Modelo de negocio}: Juego de pago (Pay-to-play)
	\item \textbf{Web}: \url{https://www.adultswim.com/games/duck-game}
\end{itemize}

\subsubsection{Capturas del videojuego}

\subsubsection{Aspectos positivos}

\subsubsection{Aspectos negativos}


\end{document}
